% !TeX spellcheck = cs_CZ
\documentclass[10pt,a4paper]{article}
\usepackage[utf8]{inputenc}
\usepackage{fullpage}
\usepackage{times}
\usepackage{sectsty}
\usepackage{listings}
\usepackage{courier}

\setlength\parindent{0pt}
\renewcommand{\lstlistingname}{Výpis}
\lstset{basicstyle=\footnotesize\ttfamily,breaklines=true}

\begin{document}
\begin{center}
    \Large{\textbf{IDS: Informační systém krevní banky}} \\
    Popis implementace a použití pokročilých objektů schématu databáze
\end{center}

\section{Triggery}
\subsection{Automatické generování hodnoty primárního klíče}
    Trigger \textbf{TRG\_AUTO\_ID\_INC} automaticky generuje hodnotu primárního
    klíče tabulky \textit{pobocka} v případě, že je hodnota primárního klíče
    při vkládání prázdná (tj. \textit{NULL}). Výsledná hodnota klíče je
    generována ze sekvence \textit{POBOCKA\_SEQ}.
\subsection{Aktualizace sloupce po vložení záznamu do tabulky}
    Tabulka \textit{darce} obsahuje sloupec \textit{posledni\_odber},
    který se musí vhodně aktualizovat po každém vložení nového záznamu
    do tabulky \textit{davka}. Cílem triggeru \textbf{UPDATE\_COLLECTION\_DATE}
    je odstranění této závislosti a tím odstranění potenciálního zdroje chyb.

    Jak již název napovídá, trigger nastaví datum a čas ve sloupci
    \textit{posledni\_odber} u patřičného dárce na aktuální, a to po každém
    vložení nového záznamu do tabulky \textit{davka}.
\section{Funkce a procedury}
\subsection{Výpočet věku dárce}
    Funkce \textbf{DONOR\_AGE} slouží pro výpočet věku dárce z jeho data
    narození (protože ukládání věku do databáze je špatný nápad). Funkce
    má jediný argument a to datum narození. Výsledkem je věk dárce zaokrouhlený
    na jedno desetinné místo. V případě, že je výsledný věk menší jako 0,
    je vyvolána výjimka s číslem \textit{-20001}.

    Chování funkce je předvedeno hned v dalším bloku, kde je pro každého dárce
    z tabulky \textit{darce} vypočítán věk a poté, společně se jménem a
    příjmením, vypsán na obrazovku. Mezi záznamy je přidán i jeden s neplatným
    datem narození, aby demonstroval vyvolání a odchycení výjimky.
\subsection{Vypočítání data možného odběru}
    Další užitečnou funkcí je \textbf{NEXT\_COLLECTION}, která z data
    posledního odběru dárce vypočítá datum následujícího možného odběru.
    Standardně se mezi odběry musí udělat minimálně tříměsíční pauza a takto
    je také tato funkce implementována. Výsledkem funkce je datum posledního
    odběru + tři měsíce v případně, že poslední odběr proběhl před méně než
    třemi měsíci. V opačném případě funkce vrací aktuální datum.

    Jednoduchá ukázka funkce se nachází v následujícím bloku, kde je datum
    dalšího možného odběru vypočítáno pro všechny dárce z tabulky
    \textit{darce}.
\section{Index}
    K nalezení místa pro ideální umístění indexu je využit příkaz
    \textbf{EXPLAIN PLAN}. Ten nám pro zadaný dotaz vytvoří novou tabulku,
    ve které nalezneme postup vyhodnocení výrazu a získání výsledných dat.

    Pro ukázku výstupu jsem vybral následující dotaz:
    \begin{lstlisting}[language=SQL,
                       caption=Dotaz pro demonstraci optimalizace]
SELECT d.jmeno, d.prijmeni, d.rodne_cislo, COUNT(o.id) AS pocet_odberu
FROM odber o
LEFT JOIN darce d
ON o.id_darce = d.id
WHERE d.id_pobocky = 1
GROUP BY d.jmeno, d.prijmeni, d.rodne_cislo
ORDER BY pocet_odberu DESC;
    \end{lstlisting}

    Zavoláním \textit{EXPLAIN PLAN FOR} společně s výše zmíněným dotazem,
    dostaneme výslednou tabulku, která po vhodném zformátování vypadá
    následovně:

    \begin{lstlisting}[caption=EXPLAIN PLAN bez zavedení indexu,
                       label={lst:ep1},xleftmargin=0.2\textwidth]
---------------------------------------------------------
| Id  | Operation                      | Name           |
---------------------------------------------------------
|   0 | SELECT STATEMENT               |                |
|   1 |  SORT ORDER BY                 |                |
|   2 |   HASH GROUP BY                |                |
|   3 |    NESTED LOOPS                |                |
|   4 |     NESTED LOOPS               |                |
|   5 |      TABLE ACCESS FULL         | ODBER          |
|   6 |      INDEX UNIQUE SCAN         | SYS_C001740260 |
|   7 |     TABLE ACCESS BY INDEX ROWID| DARCE          |
---------------------------------------------------------
    \end{lstlisting}

    Ve výsledné tabulce lze zpozorovat několik zajímavých klíčkových slov.
    Prvním z nich je \textit{NESTED LOOPS}. Tato funkce se využívá při
    spojování menších tabulek dohromady v rámci dotazu. Interní optimalizátor
    databázového systému si dané spojení rozdělí na dvě části (vnitřní a vnější
    tabulku) a pak se  pro každý řádek vnější tabulky zpracují všechny řádky
    vnitřní tabulky.

    Ve výpisu \ref{lst:ep1} je vnitřní tabulkou \textit{darce} a vnější
    tabulkou \textit{odber}. Vnější cyklus vybere všechny řádky z
    tabulky \textit{odber} a pro každý získaný řádek vybere odpovídající
    řádek z tabulky \textit{darce}. Dále se zde přistupuje jen k řádkům
    z tabulky \textit{darce}, které mají určité ID.\newline

    Jednou z možností, jak tento dotaz optimalizovat, je zavedení indexu na
    sloupce, které se používají k vlastnímu spojování tabulek. V tomto případě
    se jako jeden z kandidátu jeví sloupec \textit{id\_pobocky} z tabulky
    \textit{darce}. Po zavedení indexu je výstup EXPLAIN PLAN následující:

    \begin{lstlisting}[caption=EXPLAIN PLAN se zavedeným indexem,
                       label={lst:ep2},xleftmargin=0.19\textwidth]
-----------------------------------------------------------
| Id  | Operation                              | Name     |
-----------------------------------------------------------
|   0 | SELECT STATEMENT                       |          |
|   1 |  SORT ORDER BY                         |          |
|   2 |   HASH GROUP BY                        |          |
|   3 |    HASH JOIN                           |          |
|   4 |     TABLE ACCESS FULL                  | ODBER    |
|   5 |     TABLE ACCESS BY INDEX ROWID BATCHED| DARCE    |
|   6 |      INDEX RANGE SCAN                  | TEST_IDX |
-----------------------------------------------------------
    \end{lstlisting}

    Po zavedení indexu se dva vnořené cykly změnily na jeden
    \textit{HASH JOIN}, který se využívá při spojování větších tabulek, kdy
    se pro menší tabulku udělá v paměti hashovací tabulka, která se potom
    využívá při prohledávání větší tabulky. V optimálním případě poté stačí
    pro výsledné data projít větší tabulku jen jednou. Také přístup k řádkům
    z tabulky \textit{darce} se nyní provádí pomocí indexované hodnoty sloupce
    \textit{id\_pobocky}.

    \end{document}
