% !TeX spellcheck = cs_CZ
\documentclass[10pt,a4paper]{article}
\usepackage[utf8]{inputenc}
\usepackage{fullpage}
\usepackage{times}
\usepackage{sectsty}
\usepackage{verbatim}
\usepackage{enumitem}

\makeatletter
\newcommand{\verbatimfont}[1]{\def\verbatim@font{\fontfamily{Courier}}}
\makeatother
\setlength{\parindent}{0pt}

\begin{document}
\begin{center}
	\Large{\textbf{IPK: Přenos souborů}} \\
	\textbf{Popis aplikačního protokolu}
\end{center}
	
\section{Úvod}
	Cílem projektu bylo vytvoření aplikačního protokolu pro jednoduchý přenos souborů mezi klientem a serverem, pomocí protokolu TCP.
\section{Implementace}
	Samotná implementace se skládá z velmi jednoduchého protokolu. První požadavek zasílá klient, kde formát vypadá následovně:
	\begin{verbatim}
		IPK VER CMD FILENAME
	\end{verbatim}
	
	Jednotlivé elementy mají následující význam:
	\begin{description}[labelindent=1cm]
		\item[IPK] Označení protokolu, v našem případě je to všude právě IPK
		\item[VER] Verze protokolu, v aktuální implementaci se nijak nekontroluje
		\item[CMD] Příkaz, který serveru říká, zda-li chceme soubor nahrávat či stahovat (PUT/GET)
		\item[FILENAME] Název souboru, se kterým se bude provádět předchozí operace
	\end{description}
	
	S pomocí výše uvedených informací jsme nyní schopni sestavit skutečné příklady:
	\begin{verbatim}
		IPK 0.1 PUT image.jpg    # Nahrání souboru image.jpg na server 
		IPK 0.1 GET invoice.pdf  # Stáhnutí souboru invoice.pdf ze serveru
	\end{verbatim}
		
	\noindent Na tyto žádosti server odpovídá následující zprávou:
	\begin{verbatim}
		XX STATUS_MSG
	\end{verbatim}
	
	kde \textbf{XX} je číslo stavového kódu a \textbf{STATUS\_MSG} zpráva, která se k danému kódu vztahuje. V případě, že je kód roven nule, je žádost validní a je možno přistoupit k vlastnímu přenosu dat. Z důvodu synchronizace musí klient před začátkem odesílání/přijímání dat odeslat serveru zprávu \textbf{READY}, čímž říká, že je připraven k přenosu. V opačném případě je spojení ukončeno a klient může z výše uvedené zprávy zjistit, kde došlo k chybě.
	
	Po této krátké výměně zpráv již dochází k vlastnímu přenosu, který je ukončen jednou ze stran - v případě žádosti \textbf{GET} se jedná o server (přečtení konce souboru), v případě \textbf{PUT} jde o klienta (tentýž důvod).\newline
	
	Server také obsahuje kontrolu jmen souborů, aby klient nemohl pracovat se soubory mimo aktuální složku serveru.
	
	\section{Příklady komunikace}
	Samotný server přijímá pouze jediný parametr \textbf{-p} společně s označením portu, na kterém bude poslouchat:
	\begin{verbatim}
		./server -p 12345
	\end{verbatim}
	
	Nyní je server připraven pro požadavky klienta:
	\begin{verbatim}
		./client -h localhost -p 12345 -u myfile
		Uploading file 'myfile'
		Waiting for server
		File uploaded
	\end{verbatim}
	
	Na serverové části můžeme vidět následující výstup:
	\begin{verbatim}
		[worker] spawned
		[worker] PUT request for file 'myfile'
		[worker] waiting for client
		[worker] file saved
		[worker] quitting
	\end{verbatim}
	
	Výstup s žádostí \textbf{GET}:
	\begin{verbatim}
		KLIENT:
		./client -h localhost -p 12345 -d myfile
		Downloading file 'myfile'
		Waiting for server
		File downloaded
		
		SERVER:
		[worker] spawned
		[worker] GET request for file 'myfile'
		[worker] waiting for client
		[worker] file sent
		[worker] quitting
	\end{verbatim}
	
	Příklad zpracování neplatného souboru:
	\begin{verbatim}
		KLIENT:
		./client -h localhost -p 12345 -d ../notmyfile
		Can't GET file ../notmyfile
		Error: 4: INVALID_FILE
		An error has occured during command processing
		
		SERVER:
		[worker] spawned
		[worker] received invalid file name
		[worker] quitting
	\end{verbatim}
\end{document}